%%%%%%%%%%%%%%%%%%%%%%%%%%%%%%%%%%%%%%%%%
% Beamer Presentation
% LaTeX Template
% Version 1.0 (10/11/12)
%
% This template has been downloaded from:
% http://www.LaTeXTemplates.com
%
% License:
% CC BY-NC-SA 3.0 (http://creativecommons.org/licenses/by-nc-sa/3.0/)
%
%%%%%%%%%%%%%%%%%%%%%%%%%%%%%%%%%%%%%%%%%

%----------------------------------------------------------------------------------------
%	PACKAGES AND THEMES
%----------------------------------------------------------------------------------------

\documentclass[xcolor=table]{beamer}

\mode<presentation> {

% The Beamer class comes with a number of default slide themes
% which change the colors and layouts of slides. Below this is a list
% of all the themes, uncomment each in turn to see what they look like.

%\usetheme{default}
%\usetheme{AnnArbor}
%\usetheme{Antibes}
%\usetheme{Bergen}
%\usetheme{Berkeley}
%\usetheme{Berlin}
%\usetheme{Boadilla}
%\usetheme{CambridgeUS}
%\usetheme{Copenhagen}
%\usetheme{Darmstadt}
%\usetheme{Dresden}
%\usetheme{Frankfurt}
%\usetheme{Goettingen}
%\usetheme{Hannover}
%\usetheme{Ilmenau}
%\usetheme{JuanLesPins}
%\usetheme{Luebeck}
\usetheme{Madrid}
%\usetheme{Malmoe}
%\usetheme{Marburg}
%\usetheme{Montpellier}
%\usetheme{PaloAlto}
%\usetheme{Pittsburgh}
%\usetheme{Rochester}
%\usetheme{Singapore}
%\usetheme{Szeged}
%\usetheme{Warsaw}

% As well as themes, the Beamer class has a number of color themes
% for any slide theme. Uncomment each of these in turn to see how it
% changes the colors of your current slide theme.

%\usecolortheme{albatross}
%\usecolortheme{beaver}
%\usecolortheme{beetle}
%\usecolortheme{crane}
%\usecolortheme{dolphin}
%\usecolortheme{dove}
%\usecolortheme{fly}
%\usecolortheme{lily}
%\usecolortheme{orchid}
%\usecolortheme{rose}
%\usecolortheme{seagull}
%\usecolortheme{seahorse}
%\usecolortheme{whale}
%\usecolortheme{wolverine}

%\setbeamertemplate{footline} % To remove the footer line in all slides 
%%uncomment this line
%\setbeamertemplate{footline}[page number] % To replace the footer line in all 
%%slides with a simple slide count uncomment this line

%\setbeamertemplate{navigation symbols}{} % To remove the navigation symbols 
%%from the bottom of all slides uncomment this line
}

\usepackage{graphicx} % Allows including images
\usepackage{booktabs} % Allows the use of \toprule, \midrule and \bottomrule in 
%tables
\usepackage[french]{babel}
\usepackage{csquotes}

\usepackage[font=tiny]{subfig}

\usepackage{pgfplots}
\usepgfplotslibrary{fillbetween}
\pgfplotsset{every tick label/.append style={font=\tiny}}

\newcommand\galap[0]{\textsc{G-alap}}
\newcommand\galapllf[0]{\textsc{G-alap-LLF}}
\newcommand\galapedf[0]{\textsc{G-alap-EDF}}
\newcommand\galapezl[0]{\textsc{G-alap-EDZL}}

\setbeamertemplate{navigation symbols}{}

%----------------------------------------------------------------------------------------
%	TITLE PAGE
%----------------------------------------------------------------------------------------

\title[]{Exposé candidature 
Ingénieur d'Études et de Recherche en Systèmes Autonomes} % The short title 
%appears at the 
%bottom of every slide, the full title is only on the title page

\author{Roberto Medina} % Your name
\institute[Inria] % Your institution as it will appear on the bottom of every 
%slide, may be shorthand to save space
{
Post-doctorant dans l'équipe Kopernic, Inria\\
Docteur en informatique de Télécom ParisTech\\
 % Your institution for the title 
%page
\medskip
\textit{roberto.medina-bonilla@inria.fr} % Your email address
}
\date{3 Juillet 2019} % Date, can be changed to a custom date

\begin{document}

\begin{frame}
\titlepage % Print the title page as the first slide
\end{frame}

%----------------------------------------------------------------------------------------
%	PRESENTATION SLIDES
%----------------------------------------------------------------------------------------

\begin{frame}
	\frametitle{Curriculum Vitae}
	\textbf{2015 - Master 2 Recherche} en Informatique - 
			Université Pierre et Marie Curie (Paris 6)
		\begin{itemize}
			\item Spécialité : Systèmes et Applications Distribués
			\item Stage à Télécom ParisTech : Conception et développement de 
			systèmes critiques sur multi-c\oe{}urs.
		\end{itemize}
	
	\textbf{2019 - Doctorat} en Informatique - Télécom ParisTech
		\begin{itemize}
			\item Déploiement de systèmes à flots de données en criticité mixte 
			pour architectures multi-c\oe{}urs
			\item Soutenance en Janvier 2019.
			\item \textbf{Assistant d'enseigment} 100h/3ans - élèves 
			ingénieurs/Master/formation continue
		\end{itemize}
	\textbf{2019 - Post-doctorat} Inria, Équipe Kopernic (Paris)
		\begin{itemize}
			\item Analyse probabiliste des systèmes temps-réel avec contraintes 
			énergétiques
		\end{itemize}
\end{frame}

%-------------------------------------------------------------------------------

\begin{frame}
	\frametitle{Thème de recherche}
	\centering
	\textbf{Optimisation des techniques d'ordonnancement pour des systèmes 
	embarqués critiques}
	\begin{itemize}
		\item Criticité mixte sur architectures multi-c\oe{}urs 
		avec tâches dépendantes.
		\begin{itemize}
			\item Calcul de tables d'ordonnancement pour les différents modes 
			d'exécutions du système.
			\item Quantification et amélioration de la disponibilité des tâches 
			moins critiques.
		\end{itemize}
		\item Contraintes énergétiques pour des systèmes temps-réel.
		\begin{itemize}
			\item Mise en évidence des limites des approches existantes.
		\end{itemize}
	\end{itemize}
\end{frame}

%-------------------------------------------------------------------------------
\section{Ordonnancement à criticité mixte pour tâches dépendantes}
%-------------------------------------------------------------------------------
\subsection{Modèle d'exécution}
%-------------------------------------------------------------------------------

\begin{frame}
	\frametitle{Pire temps d'exécution (WCET)}
	\begin{figure}
		\centering
		\includegraphics<1|handout:0>[width=10cm]{figs/clo0.pdf}
		\includegraphics<2|handout:0>[width=10cm]{figs/clo1.pdf}
		\includegraphics<3|handout:0>[width=10cm]{figs/clo2.pdf}
		\includegraphics<4>[width=10cm]{figs/clo.pdf}
	\end{figure}

	\begin{itemize}
	\item Estimer le WCET : problème très difficile.
		\begin{itemize}
			\item<3-4> Plusieures méthodes possibles pour obtenir une 
			estimation.
			\item<4> Très souvent borné.
			\item<4> Tâches s'exécutent rarement jusqu'au WCET.
			\item<4> Architectures muti-c\oe{}urs peu predictibles.
		\end{itemize}
	\end{itemize}
\end{frame}

%-------------------------------------------------------------------------------

\begin{frame}
	\frametitle{Le modèle à criticité-mixte}
	\begin{enumerate}
		
		\item Incorporer des tâches avec différents niveaux de criticité : HI et LO.
		\item Modes d'exécution :
		\begin{itemize}
			\item Mode LO-criticality : Tâches HI  + tâches LO .
			\item Mode HI-criticality : Tâches HI  $\rightarrow$ Tâches LO dégradées.
		\end{itemize}
		\item Plusieurs pire temps d'exécution.
		\begin{itemize}
			\item $C_i(LO)$: Max. temps d'exécution observé (system designers).
			\item $C_i(HI)$: Borne sup. temps d'exécution (static analysis).
		\end{itemize}
		\item Système passe dans le mode HI-criticality quand c'est nécessaire.
	\end{enumerate}	
\end{frame}

%-------------------------------------------------------------------------------

\begin{frame}
	\frametitle{Limites des approches existantes}
\end{frame}

\begin{frame}
	\frametitle{Mixed-Criticality Directed Acyclic Graphs}
	\begin{columns}
		\begin{column}{0.4\textwidth}
			MC-DAG $G \in \mathcal{G}$.\\
			$G=(V, E, D, T)$.
			\begin{itemize}
				\item<2-> $V$ ensemble de n\oe{}uds.
				\item<3-> $E \subseteq (V \times V)$ ensemble d'arcs.
				\item<4-> $D$ échéance.
				\item<5-> $T$ période.
			\end{itemize}
			\only<6->{N\oe{}uds $\tau_i = (\chi_i, 
			C_i(\chi_1), \dots, C_i(\chi_\ell))$}
			\begin{itemize}
				\item<6-> $\chi_i \in \mathcal{CL}$ niveau de criticité.
				\item<7-> $C_i(\chi_1), \dots, C_i(\chi_\ell)$ pire temps d'exécution.
			\end{itemize}
		\end{column}
		\begin{column}{0.6\textwidth}
			\begin{figure}
				\includegraphics<1|handout:0>[width=6cm]{figs/multidag_lo0.pdf}
				\includegraphics<2|handout:0>[width=6cm]{figs/multidag_lo1.pdf}
				\includegraphics<3|handout:0>[width=6cm]{figs/multidag_lo2.pdf}
				\includegraphics<4|handout:0>[width=6cm]{figs/multidag_lo3.pdf}
				\includegraphics<5|handout:0>[width=6cm]{figs/multidag_lo4.pdf}
				\includegraphics<6|handout:0>[width=6cm]{figs/multidag_lo5.pdf}
				\includegraphics<7>[width=6cm]{figs/multidag_lo.pdf}
			\end{figure}
		\end{column}
	\end{columns}
\end{frame}

%------------------------------------------------------------------------------
\subsection{Outil de génération}
%------------------------------------------------------------------------------

\begin{frame}
	\frametitle{Unbiased generation tool}		
	\begin{itemize}
		\item Integrate concepts of different communities.
		\begin{itemize}
			\item Avoid particular DAG shapes: create vertices using 
			layers.
			\item Distribute system utilization 
			uniformly.
		\end{itemize}
		\item Many parameters need to be taken into account.
		\begin{itemize}
			\item Utilization of the system.
			\item Number of MC-DAGs.
			\item Number of vertices.
			\item Probability to have an edge.
			\item Ratio HI/LO-criticality tasks.
		\end{itemize}
		\item Integration into an open-source framework~\footnote{MC-DAG 
			framework - \url{https://github.com/robertoxmed/MC-DAG}}.
	\end{itemize}
\end{frame}

%------------------------------------------------------------------------------

\begin{frame}
	\frametitle{Unbiased generation algorithm}
	
	\begin{columns}
		\begin{column}{.6\textwidth}
			\begin{exampleblock}{MC-DAG generator}
				\begin{enumerate}
					\item<1-> Distribute utilization.
					\item<1-> Draw a random deadline.
					\item<2-> For each criticality level:
					\begin{itemize}
						\item<3-> \textbf{\textit{Vertex generation phase}} by 
						layers.
						\item<4-> \textbf{\textit{Incorporation of edges}}
						following a given probability.
						\item<5-> \textbf{\textit{Reduction phase}} reach a 
						reduction factor.
					\end{itemize}
				\end{enumerate}
			\end{exampleblock}
		\end{column}
		
		\begin{column}{0.4\textwidth}
			\begin{figure}
				\includegraphics<3|handout:0>[width=4.5cm]{figs/random_dag0.pdf}
				\includegraphics<4|handout:0>[width=4.5cm]{figs/random_dag1.pdf}
				\includegraphics<5|handout:0>[width=4.5cm]{figs/random_dag2.pdf}
				\includegraphics<6->[width=4.5cm]{figs/random_dag.pdf}
			\end{figure}
		\end{column}
	\end{columns}
\end{frame}

%------------------------------------------------------------------------------

\begin{frame}
	\frametitle{Experimental results}
	
	\begin{figure}
		\subfloat[\textsc{G-LLF, $|\mathcal{G}| = 1$}]{
			\begin{tikzpicture}[every mark/.append style={mark size=1pt}]
			\begin{axis}[height=4cm, width=0.35\textwidth,
			legend style={nodes={scale=0.4, transform shape}},
			legend pos=north east,grid,
			log ticks with fixed point,
			ytick={0, 0.1, ..., 1.1},
			xtick={0.2, 0.3, ..., 1.1},
			xmin=0.25,	xmax=1,cycle list name=exotic,
			ymin=0]
			
			\path[name path=axis] (axis cs:0.25,0) -- (axis cs:1,0);
			
			\addplot+[name path=llf2] table [x=Unorm, y=LLF] 
			{data/sched/sched_l2_d1.csv};
			\addplot+[name path=llf4] table [x=Unorm, y=LLF] 
			{data/sched/sched_l4_d1.csv};
			\addplot+[name path=llf5] table [x=Unorm, y=LLF] 
			{data/sched/sched_l5_d1.csv};
			
			\addplot[name path=fedllf2] 	table 	[x=Unorm, y=FEDLLF] 
			{data/sched/sched_l2_d1.csv};
			\addplot+[name path=fedllf4] table [x=Unorm, y=FEDLLF] 
			{data/sched/sched_l4_d1.csv};
			\addplot+[name path=fedllf5] table 	[x=Unorm, y=FEDLLF] 
			{data/sched/sched_l5_d1.csv};
			
			\pgfplotsset{cycle list shift=-6}
			\addplot +[fill opacity=0.5] fill between [of=llf2 and axis];
			\addplot +[fill opacity=0.5] fill between [of=llf4 and axis];
			\addplot +[fill opacity=0.5] fill between [of=llf5 and axis];
			\addplot +[gray,fill opacity=0.5] fill between [of=fedllf2 and axis];
			\addplot +[fill opacity=0.5] fill between [of=fedllf4 and axis];
			\addplot +[fill opacity=0.5] fill between [of=fedllf5 and axis];
			
			\end{axis}
			\end{tikzpicture}
			\label{subfig:acceptance-llf}
		}
		\subfloat[\textsc{G-EDF, $|\mathcal{G}| = 1$}]{
			\begin{tikzpicture}[every mark/.append style={mark size=1.2pt}]
			\begin{axis}[	height=4cm, width=0.35\textwidth,
			legend style={nodes={scale=0.5, transform shape}},
			legend pos=north east,grid,
			log ticks with fixed point,
			ytick={0, 0.1, ..., 1.1},
			xtick={0.2, 0.3, ..., 1.1},
			xmin=0.25,	xmax=1,cycle list name=exotic,
			ymin=0]
			
			\path[name path=axis] (axis cs:0.25,0) -- (axis cs:1,0);
			
			\addplot+[name path=edf2] table [x=Unorm, y=EDF] 
			{data/sched/sched_l2_d1.csv};
			\addplot+[name path=edf4] table [x=Unorm, y=EDF] 
			{data/sched/sched_l4_d1.csv};
			\addplot+[name path=edf5] table [x=Unorm, y=EDF] 
			{data/sched/sched_l5_d1.csv};
			
			\addplot[name path=fededf2] 	table 	[x=Unorm, y=FEDEDF] 
			{data/sched/sched_l2_d1.csv};
			\addplot+[name path=fededf4] table [x=Unorm, y=FEDEDF] 
			{data/sched/sched_l4_d1.csv};
			\addplot+[name path=fededf5] table 	[x=Unorm, y=FEDEDF] 
			{data/sched/sched_l5_d1.csv};
			
			\pgfplotsset{cycle list shift=-6}
			\addplot +[fill opacity=0.5] fill between [of=edf2 and axis];
			\addplot +[fill opacity=0.5] fill between [of=edf4 and axis];
			\addplot +[fill opacity=0.5] fill between [of=edf5 and axis];
			\addplot +[gray,fill opacity=0.5] fill between [of=fededf2 and axis];
			\addplot +[fill opacity=0.5] fill between [of=fededf4 and axis];
			\addplot +[fill opacity=0.5] fill between [of=fededf5 and axis];
			
%			\addlegendentry{\galap, $|\mathcal{CL}| = 2$}
%			\addlegendentry{\galap, $|\mathcal{CL}| = 4$}
%			\addlegendentry{\galap, $|\mathcal{CL}| = 5$}
%			\addlegendentry{\textsc{Fed}, $|\mathcal{CL}| = 2$}
%			\addlegendentry{\textsc{Fed}, $|\mathcal{CL}| = 4$}
%			\addlegendentry{\textsc{Fed}, $|\mathcal{CL}| = 5$}
			
			\end{axis}
			\end{tikzpicture}
			\label{subfig:acceptance-edf-d1}
		}
		\subfloat[\textsc{G-EDZL, $|\mathcal{G}| = 1$}]{
			\begin{tikzpicture}[every mark/.append style={mark size=1.2pt}]
			\begin{axis}[	height=4cm, width=0.35\textwidth,
			legend style={nodes={scale=0.4, transform shape}},
			legend pos=north east,grid,
			log ticks with fixed point,
			ytick={0, 0.1, ..., 1.1},
			xtick={0.2, 0.3, ..., 1.1},
			xmin=0.25,	xmax=1,cycle list name=exotic,
			ymin=0]
			\path[name path=axis] (axis cs:0.25,0) -- (axis cs:1,0);
			
			\addplot+[name path=ezl2] table [x=Unorm, y=EZL] 
			{data/sched/sched_l2_d1.csv};
			\addplot+[name path=ezl4] table [x=Unorm, y=EZL] 
			{data/sched/sched_l4_d1.csv};
			\addplot+[name path=ezl5] table [x=Unorm, y=EZL] 
			{data/sched/sched_l5_d1.csv};
			
			\addplot[name path=fedezl2] 	table 	[x=Unorm, y=FEDEZL] 
			{data/sched/sched_l2_d1.csv};
			\addplot+[name path=fedezl4] table [x=Unorm, y=FEDEZL] 
			{data/sched/sched_l4_d1.csv};
			\addplot+[name path=fedzl5] table 	[x=Unorm, y=FEDEZL] 
			{data/sched/sched_l5_d1.csv};
			
			\pgfplotsset{cycle list shift=-6}
			\addplot +[fill opacity=0.5] fill between [of=ezl2 and axis];
			\addplot +[fill opacity=0.5] fill between [of=ezl4 and axis];
			\addplot +[fill opacity=0.5] fill between [of=ezl5 and axis];
			\addplot +[gray,fill opacity=0.5] fill between [of=fedezl2 and axis];
			\addplot +[fill opacity=0.5] fill between [of=fedezl4 and axis];
			\addplot +[fill opacity=0.5] fill between [of=fedzl5 and axis];
			
			\end{axis}
			\end{tikzpicture}
			\label{subfig:acceptance-ezl-d1}
		}
	
			\subfloat[\textsc{G-LLF, $|\mathcal{G}| = 2$}]{
				\begin{tikzpicture}[every mark/.append style={mark size=1.2pt}]
				\begin{axis}[	height=4cm, width=0.35\textwidth,
				legend style={nodes={scale=0.4, transform shape}},
				legend pos=north east,grid,
				log ticks with fixed point,
				ytick={0, 0.1, ..., 1.1},
				xtick={0.2, 0.3, ..., 1.1},
				xmin=0.25,	xmax=1,cycle list name=exotic,
				ymin=0]
				\path[name path=axis] (axis cs:0.25,0) -- (axis cs:1,0);
				
				\addplot+[name path=llf2] table [x=Unorm, y=LLF] 
				{data/sched/sched_l2_d2.csv};
				\addplot+[name path=llf4] table [x=Unorm, y=LLF] 
				{data/sched/sched_l4_d2.csv};
				\addplot+[name path=llf5] table [x=Unorm, y=LLF] 
				{data/sched/sched_l5_d2.csv};
				
				\addplot[name path=fedllf2] 	table 	[x=Unorm, y=FEDLLF] 
				{data/sched/sched_l2_d2.csv};
				\addplot+[name path=fedllf4] table [x=Unorm, y=FEDLLF] 
				{data/sched/sched_l4_d2.csv};
				\addplot+[name path=fedllf5] table 	[x=Unorm, y=FEDLLF] 
				{data/sched/sched_l5_d2.csv};
				
				\pgfplotsset{cycle list shift=-6}
				\addplot +[fill opacity=0.5] fill between [of=llf2 and axis];
				\addplot +[fill opacity=0.5] fill between [of=llf4 and axis];
				\addplot +[fill opacity=0.5] fill between [of=llf5 and axis];
				\addplot +[gray,fill opacity=0.5] fill between [of=fedllf2 and axis];
				\addplot +[fill opacity=0.5] fill between [of=fedllf4 and axis];
				\addplot +[fill opacity=0.5] fill between [of=fedllf5 and axis];
				\end{axis}
				\end{tikzpicture}
				\label{subfig:acceptance-llf-d2}
			}
			\subfloat[\textsc{G-EDF, $|\mathcal{G}| = 2$}]{
				\begin{tikzpicture}[every mark/.append style={mark size=1.2pt}]
				\begin{axis}[	height=4cm, width=0.35\textwidth,
				legend style={nodes={scale=0.4, transform shape}},
				legend pos=north east,grid,
				log ticks with fixed point,
				ytick={0, 0.1, ..., 1.1},
				xtick={0.2, 0.3, ..., 1.1},
				xmin=0.25,	xmax=1,cycle list name=exotic,
				ymin=0]
				\path[name path=axis] (axis cs:0.25,0) -- (axis cs:1,0);
				
				\addplot+[name path=edf2] table [x=Unorm, y=EDF] 
				{data/sched/sched_l2_d2.csv};
				\addplot+[name path=edf4] table [x=Unorm, y=EDF] 
				{data/sched/sched_l4_d2.csv};
				\addplot+[name path=edf5] table [x=Unorm, y=EDF] 
				{data/sched/sched_l5_d2.csv};
				
				\addplot[name path=fededf2] 	table 	[x=Unorm, y=FEDEDF] 
				{data/sched/sched_l2_d2.csv};
				\addplot+[name path=fededf4] table [x=Unorm, y=FEDEDF] 
				{data/sched/sched_l4_d2.csv};
				\addplot+[name path=fededf5] table 	[x=Unorm, y=FEDEDF] 
				{data/sched/sched_l5_d2.csv};
				
				\pgfplotsset{cycle list shift=-6}
				\addplot +[fill opacity=0.5] fill between [of=edf2 and axis];
				\addplot +[fill opacity=0.5] fill between [of=edf4 and axis];
				\addplot +[fill opacity=0.5] fill between [of=edf5 and axis];
				\addplot +[gray,fill opacity=0.5] fill between [of=fededf2 and axis];
				\addplot +[fill opacity=0.5] fill between [of=fededf4 and axis];
				\addplot +[fill opacity=0.5] fill between [of=fededf5 and axis];
				\end{axis}
				\end{tikzpicture}
				\label{subfig:acceptance-edf}
			}
			\subfloat[\textsc{G-EDZL, $|\mathcal{G}| = 2$}]{
				\begin{tikzpicture}[every mark/.append style={mark size=1.2pt}]
				\begin{axis}[	height=4cm, width=0.35\textwidth,
				legend style={nodes={scale=0.4, transform shape}},
				legend pos=north east,grid,
				log ticks with fixed point,
				ytick={0, 0.1, ..., 1.1},
				xtick={0.2, 0.3, ..., 1.1},
				xmin=0.25,	xmax=1,cycle list name=exotic,
				ymin=0]
				
				\path[name path=axis] (axis cs:0.25,0) -- (axis cs:1,0);
				
				\addplot+[name path=ezl2] table [x=Unorm, y=EZL] 
				{data/sched/sched_l2_d2.csv};
				\addplot+[name path=ezl4] table [x=Unorm, y=EZL] 
				{data/sched/sched_l4_d2.csv};
				\addplot+[name path=ezl5] table [x=Unorm, y=EZL] 
				{data/sched/sched_l5_d2.csv};
				
				\addplot[name path=fedezl2] 	table 	[x=Unorm, y=FEDEZL] 
				{data/sched/sched_l2_d2.csv};
				\addplot+[name path=fedezl4] table [x=Unorm, y=FEDEZL] 
				{data/sched/sched_l4_d2.csv};
				\addplot+[name path=fedzl5] table 	[x=Unorm, y=FEDEZL] 
				{data/sched/sched_l5_d2.csv};
				
				\pgfplotsset{cycle list shift=-6}
				\addplot +[fill opacity=0.5] fill between [of=ezl2 and axis];
				\addplot +[fill opacity=0.5] fill between [of=ezl4 and axis];
				\addplot +[fill opacity=0.5] fill between [of=ezl5 and axis];
				\addplot +[gray,fill opacity=0.5] fill between [of=fedezl2 and axis];
				\addplot +[fill opacity=0.5] fill between [of=fedezl4 and axis];
				\addplot +[fill opacity=0.5] fill between [of=fedzl5 and axis];
				
				\end{axis}
				\end{tikzpicture}
				\label{subfig:acceptance-ezl-d2}
			}
	\end{figure}
\end{frame}

%\begin{frame}
%	\frametitle{Expe}
%	
%	\begin{figure}
%		
%%%%%%%%%%%%%%%%%%%%%%%%%%%%%%%%%%%%%%%%%%%%%%%%%%%%%%%%%%%%%%%%%%%%%%%%%%%%%%%%%%%%%%%%%%%%%%%%%%%%%%%%%%%%%%%%%%%%%%%%%%%%%%%%%%%%%%%%%%%%%%%%%%%%%%%%%%%%%%%%%%%%%%%%%%%%%%%%%%%%%%%%%%%%%%%%%
%		
%%%%%%%%%%%%%%%%%%%%%%%%%%%%%%%%%%%%%%%%%%%%%%%%%%%%%%%%%%%%%%%%%%%%%%%%%%%%%%%%%%%%%%%%%%%%%%%%%%%%%%%%%%%%%%%%%%%%%%%%%%%%%%%%%%%%%%%%%%%%%%%%%%%%%%%%%%%%%%%%%%%%%%%%%%%%%%%%%%%%%%%%%%%%%%%%%
%		
%		\subfloat[\textsc{G-LLF, $|\mathcal{G}| = 2$}]{
%			\begin{tikzpicture}[every mark/.append style={mark size=1.2pt}]
%			\begin{axis}[	height=6cm, width=0.35\textwidth,
%			legend style={nodes={scale=0.4, transform shape}},
%			legend pos=north east,grid,
%			log ticks with fixed point,
%			ytick={0, 0.1, ..., 1.1},
%			xtick={0.2, 0.3, ..., 1.1},
%			xmin=0.25,	xmax=1,cycle list name=exotic,
%			ymin=0]
%			\path[name path=axis] (axis cs:0.25,0) -- (axis cs:1,0);
%			
%			\addplot+[name path=llf2] table [x=Unorm, y=LLF] 
%			{data/sched/sched_l2_d2.csv};
%			\addplot+[name path=llf4] table [x=Unorm, y=LLF] 
%			{data/sched/sched_l4_d2.csv};
%			\addplot+[name path=llf5] table [x=Unorm, y=LLF] 
%			{data/sched/sched_l5_d2.csv};
%			
%			\addplot[name path=fedllf2] 	table 	[x=Unorm, y=FEDLLF] 
%			{data/sched/sched_l2_d2.csv};
%			\addplot+[name path=fedllf4] table [x=Unorm, y=FEDLLF] 
%			{data/sched/sched_l4_d2.csv};
%			\addplot+[name path=fedllf5] table 	[x=Unorm, y=FEDLLF] 
%			{data/sched/sched_l5_d2.csv};
%			
%			\pgfplotsset{cycle list shift=-6}
%			\addplot +[fill opacity=0.5] fill between [of=llf2 and axis];
%			\addplot +[fill opacity=0.5] fill between [of=llf4 and axis];
%			\addplot +[fill opacity=0.5] fill between [of=llf5 and axis];
%			\addplot +[gray,fill opacity=0.5] fill between [of=fedllf2 and axis];
%			\addplot +[fill opacity=0.5] fill between [of=fedllf4 and axis];
%			\addplot +[fill opacity=0.5] fill between [of=fedllf5 and axis];
%			\end{axis}
%			\end{tikzpicture}
%			\label{subfig:acceptance-llf-d2}
%		}
%		\subfloat[\textsc{G-EDF, $|\mathcal{G}| = 2$}]{
%			\begin{tikzpicture}[every mark/.append style={mark size=1.2pt}]
%			\begin{axis}[	height=6cm, width=0.35\textwidth,
%			legend style={nodes={scale=0.4, transform shape}},
%			legend pos=north east,grid,
%			log ticks with fixed point,
%			ytick={0, 0.1, ..., 1.1},
%			xtick={0.2, 0.3, ..., 1.1},
%			xmin=0.25,	xmax=1,cycle list name=exotic,
%			ymin=0]
%			\path[name path=axis] (axis cs:0.25,0) -- (axis cs:1,0);
%			
%			\addplot+[name path=edf2] table [x=Unorm, y=EDF] 
%			{data/sched/sched_l2_d2.csv};
%			\addplot+[name path=edf4] table [x=Unorm, y=EDF] 
%			{data/sched/sched_l4_d2.csv};
%			\addplot+[name path=edf5] table [x=Unorm, y=EDF] 
%			{data/sched/sched_l5_d2.csv};
%			
%			\addplot[name path=fededf2] 	table 	[x=Unorm, y=FEDEDF] 
%			{data/sched/sched_l2_d2.csv};
%			\addplot+[name path=fededf4] table [x=Unorm, y=FEDEDF] 
%			{data/sched/sched_l4_d2.csv};
%			\addplot+[name path=fededf5] table 	[x=Unorm, y=FEDEDF] 
%			{data/sched/sched_l5_d2.csv};
%			
%			\pgfplotsset{cycle list shift=-6}
%			\addplot +[fill opacity=0.5] fill between [of=edf2 and axis];
%			\addplot +[fill opacity=0.5] fill between [of=edf4 and axis];
%			\addplot +[fill opacity=0.5] fill between [of=edf5 and axis];
%			\addplot +[gray,fill opacity=0.5] fill between [of=fededf2 and axis];
%			\addplot +[fill opacity=0.5] fill between [of=fededf4 and axis];
%			\addplot +[fill opacity=0.5] fill between [of=fededf5 and axis];
%			\end{axis}
%			\end{tikzpicture}
%			\label{subfig:acceptance-edf}
%		}
%		\subfloat[\textsc{G-EDZL, $|\mathcal{G}| = 2$}]{
%			\begin{tikzpicture}[every mark/.append style={mark size=1.2pt}]
%			\begin{axis}[	height=6cm, width=0.35\textwidth,
%			legend style={nodes={scale=0.4, transform shape}},
%			legend pos=north east,grid,
%			log ticks with fixed point,
%			ytick={0, 0.1, ..., 1.1},
%			xtick={0.2, 0.3, ..., 1.1},
%			xmin=0.25,	xmax=1,cycle list name=exotic,
%			ymin=0]
%			
%			\path[name path=axis] (axis cs:0.25,0) -- (axis cs:1,0);
%			
%			\addplot+[name path=ezl2] table [x=Unorm, y=EZL] 
%			{data/sched/sched_l2_d2.csv};
%			\addplot+[name path=ezl4] table [x=Unorm, y=EZL] 
%			{data/sched/sched_l4_d2.csv};
%			\addplot+[name path=ezl5] table [x=Unorm, y=EZL] 
%			{data/sched/sched_l5_d2.csv};
%			
%			\addplot[name path=fedezl2] 	table 	[x=Unorm, y=FEDEZL] 
%			{data/sched/sched_l2_d2.csv};
%			\addplot+[name path=fedezl4] table [x=Unorm, y=FEDEZL] 
%			{data/sched/sched_l4_d2.csv};
%			\addplot+[name path=fedzl5] table 	[x=Unorm, y=FEDEZL] 
%			{data/sched/sched_l5_d2.csv};
%			
%			\pgfplotsset{cycle list shift=-6}
%			\addplot +[fill opacity=0.5] fill between [of=ezl2 and axis];
%			\addplot +[fill opacity=0.5] fill between [of=ezl4 and axis];
%			\addplot +[fill opacity=0.5] fill between [of=ezl5 and axis];
%			\addplot +[gray,fill opacity=0.5] fill between [of=fedezl2 and axis];
%			\addplot +[fill opacity=0.5] fill between [of=fedezl4 and axis];
%			\addplot +[fill opacity=0.5] fill between [of=fedzl5 and axis];
%			
%			\end{axis}
%			\end{tikzpicture}
%			\label{subfig:acceptance-ezl-d2}
%		}
%	\end{figure}
%\end{frame}

%-------------------------------------------------------------------------------
\begin{frame}
	\frametitle{Publications \& Valorisation}
	\begin{table}[]
		\begin{tabular}{|l|c|l|}
			\hline
			\rowcolor[HTML]{DAE8FC} 
			Total                      & 7          &                          \\ \hline
			Journaux                   & \textit{1} & IEEE Trans. on Computers \\ \hline
			Conférences (Papiers)      & 3          & RTSS, DATE, Ada Europe   \\ \hline
			Conférences (WIP + Poster) & \textit{2} & RTSS, SIES               \\ \hline
			Thèse                      & 1          &                          \\ \hline
		\end{tabular}
	\end{table}
	\textbf{Séminaires externes et diffusion de l'information scientifique}
	\begin{itemize}
        \item INRIA, Rennes, 2017, Scheduling of DAGs for mixed-criticality systems.
		\item INRIA, Paris, 2018, Scheduling and availability analysis for DAGs on 
		mixed-criticality systems.
		\item CEA List, Palaiseau, 2018, Scheduling of multi-periodic DAGs on 
		multi-core for mixed-criticality.
		\item T\'{e}l\'{e}com ParisTech, 2018, ``Journ\'{e}es Recherche du LTCI''.
	\end{itemize}
\end{frame}



%-------------------------------------------------------------------------------
\begin{frame}
	\frametitle{Integration DTIS/SEAS: projet de recherche}
	
	\begin{itemize}
		\item \textbf{Contexte équipe :}
		\begin{itemize}
			\item Spécification, conception, développement et analyse de systèmes autonomes.
		\end{itemize}
		\item \textbf{Proposition:}
		\begin{itemize}
			\item Adaptation du comportement des systèmes autonomes en prennant en compte.
		\end{itemize}
	\end{itemize}
	\begin{columns}
		\begin{column}{.45\textwidth}
			\textbf{Thème 1 :}
		\end{column}
		\begin{column}{.45\textwidth}
			\textbf{Thème 2 :} 
		\end{column}
	\end{columns}
	
\end{frame}

%-------------------------------------------------------------------------------
\begin{frame}
	\frametitle{Integration DTIS/SEAS: compétences}
		
	\begin{displayquote}
		\begin{itemize}
			\item \textcolor[rgb]{0.2,0.5,0.2}{Formal specification languages}
			\item \textcolor[rgb]{0.2,0.5,0.2}{Languages for modeling system architectures or their behavior}
			\item \textcolor[rgb]{0.2,0.5,0.2}{Formal verification} or \textcolor[rgb]{0.2,0.5,0.2}{real-time 
			analysis techniques}
			\item Architectures and algorithms for decision-making, from task planning to execution management
			\item Decision-making and multi-machine mission management architectures, distributed between 
			ground stations and onboard systems
			\item \textcolor[rgb]{0.2,0.5,0.2}{Tools and methods for installation on physical platforms}
		\end{itemize}
	\end{displayquote}

	Repris du site \url{https://www.onera.fr/en/dtis/research-units}
	
\end{frame}

%-------------------------------------------------------------------------------
\begin{frame}
	\frametitle{Bilan de candidature}
	
	\begin{itemize}
		\item \textbf{Intégration DTIS/SEAS}
	\end{itemize}
\end{frame}

\end{document} 